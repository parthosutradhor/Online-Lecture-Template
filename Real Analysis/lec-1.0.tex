% ===============================
% Worksheet / Manual Template
% ===============================
\documentclass[12pt, a4paper]{article}

% ===============================
% Metadata
% ===============================
\newcommand*{\CourseCode}{MAT221}
\newcommand*{\CourseTitle}{Real Analysis}
\newcommand*{\Chapter}{Constructing the Rationals}
\newcommand*{\Topics}{Equivalence Relation and Constructing the Rational Numbers}

\newcommand{\ReferenceList}{%
\begin{enumerate}[leftmargin=2em, label={[\arabic{*}]}]
	\item Stephen Abbott, \textit{Understanding Analysis}, 2nd Edition, Springer, 2015.

	\item Terence Tao, \textit{Analysis I}, 3rd Edition, Texts and Readings in Mathematics, Hindustan Book Agency, 2016.
\end{enumerate}
}

\newcommand*{\University}{BRAC University}
\newcommand*{\Department}{Department of Mathematics and Natural Sciences}
\newcommand*{\LogoPath}{Brac_University_Logo} % logo filename (no extension)
\newcommand*{\DocDate}{\today}
\newcommand*{\Instructor}{Partho Sutra Dhor}
\newcommand*{\Role}{Lecturer, BRAC University, Dhaka-1212}

% ---------- Encoding & Fonts ----------
\usepackage[T1]{fontenc}
\usepackage[utf8]{inputenc}
\usepackage{lmodern}
\usepackage{microtype}
\IfFileExists{qrcode.sty}{\usepackage{qrcode}}{}

% ---------- Icons & Colors ----------
\usepackage{fontawesome5} % icons
\usepackage[dvipsnames, table]{xcolor}
\definecolor{YTred}{RGB}{255,0,0}
\definecolor{BannerLeft}{RGB}{30,64,175}
\definecolor{BannerRight}{RGB}{56,189,248}
\definecolor{Accent}{RGB}{20,120,200}

% ---------- Page Setup ----------
\usepackage[top=0.6in, bottom=1.0in, left=0.7in, right=0.7in]{geometry}
\usepackage{setspace}
\setstretch{1.2}
\setlength{\parindent}{0pt}
\setlength{\parskip}{6pt}
\pagestyle{empty}

% ---------- Math & Graphics ----------
\usepackage{amsmath, amssymb, amsfonts, mathtools, bm}
\usepackage{graphicx}
\usepackage{tikz, pgfplots}
\pgfplotsset{compat=1.18}
\usetikzlibrary{calc}

% ---------- Tables & Lists ----------
\usepackage{array, booktabs, multirow, colortbl, float}
\renewcommand{\arraystretch}{1.25}
\usepackage{enumitem}
\setlist{leftmargin=*,itemsep=2pt}

% ---------- Boxes & Links ----------
\usepackage[most]{tcolorbox}
\tcbuselibrary{breakable,skins}
\usepackage[
	colorlinks=true,
	linkcolor=MidnightBlue,
	citecolor=OliveGreen,
	urlcolor=RoyalBlue
]{hyperref}
\usepackage[nameinlink, noabbrev]{cleveref}

% ===============================
% Cover Page (styled with icons)
% ===============================
\newcommand{\FrontPage}{%
\begin{titlepage}
	\centering

	% --- Gradient banner across the top ---
	\begin{tikzpicture}[remember picture, overlay]
		\node[anchor=north west,inner sep=0pt] at ($(current page.north west)$) {%
		\begin{tikzpicture}\path[shade, left color=BannerLeft, right color=BannerRight] (0,0) rectangle (\paperwidth,1.8cm);\end{tikzpicture} };
	\end{tikzpicture}
	\vspace*{1.0cm}

	% --- Logo & University Name ---
	\includegraphics[width=0.18\textwidth]{\LogoPath}
	\par
	\vspace{6pt}
	{\Large\sffamily\bfseries \University\par} {\large\sffamily \Department\par}

	% --- Title block ---
	\vspace{24pt}
	{\normalsize\sffamily\scshape Lecture on\par}
	\vspace{6pt}
	{\Large\sffamily\bfseries \CourseTitle\ (\CourseCode)\par}

	\vspace{18pt}
	{\Huge\sffamily\bfseries \textcolor{Accent}{\Chapter}\par}
	\vspace{10pt}
	{\Large\sffamily\bfseries \textcolor{teal}{\Topics}\par}
	\vfill
	\vspace{14pt}
	\rule{0.55\textwidth}{0.8pt}\par
	\vspace{10pt}
	\vfill
	{\large\sffamily\scshape \DocDate\par}
	\vfill
	\vspace{10pt}
	{\large\sffamily\scshape Conducted by\par}
	{\LARGE\sffamily\bfseries \Instructor\par} {\large\sffamily \Role\par}

	% --- Contact / Links with icons ---
	\vspace{18pt}
	\begin{tcolorbox}
		[ colback=gray!3!white, colframe=gray!40!black, boxrule=0.6pt, arc=2pt, left=8pt,right=8pt,top=8pt,bottom=8pt,
		width=0.82\textwidth, enhanced ]
		\centering
		\sffamily {\large \faEnvelope\;\href{mailto:partho.dhor@bracu.ac.bd}{partho.dhor@bracu.ac.bd} \quad\textbar\quad \faEnvelope\;\href{mailto:parthosutradhor@gmail.com}{parthosutradhor@gmail.com} \\[4pt] For updates subscribe on \quad \textcolor{YTred}{\faYoutube}\; \href{https://www.youtube.com/@ParthoSutraDhor}{@ParthoSutraDhor} }
	\end{tcolorbox}

	\vspace{10pt}
	\begin{tikzpicture}[remember picture, overlay]
		\node[anchor=south west,inner sep=0pt] at ($(current page.south west)$) {%
		\begin{tikzpicture}\path[shade, left color=BannerRight, right color=BannerLeft] (0,0) rectangle (\paperwidth,0.9cm);\end{tikzpicture} };
	\end{tikzpicture}
\end{titlepage}%
}

% ===============================
% End Page
% ===============================
\newcommand{\EndPage}{%
\clearpage
\thispagestyle{empty}
\begin{tikzpicture}[remember picture, overlay]
	\node[anchor=north west,inner sep=0pt] at ($(current page.north west)$) {%
	\begin{tikzpicture}\path[shade, left color=BannerLeft, right color=BannerRight] (0,0) rectangle (\paperwidth,1.2cm);\end{tikzpicture} };
\end{tikzpicture}

\vspace*{0.8cm}
\begin{center}
	{\fontsize{38}{42}\selectfont\sffamily\bfseries Thank You!\par}
	\vspace{0.4em}
	{\large\sffamily We’d love your questions and feedback.\par}
	\vspace{1.2em}
	\rule{0.5\textwidth}{0.8pt}\par
	\vspace{1.2em}

	\begin{tcolorbox}
		[ colback=gray!2!white, colframe=gray!35!black, boxrule=0.7pt, arc=3pt,
		width=0.86\textwidth, left=10pt,right=10pt,top=10pt,bottom=10pt, enhanced ]
		\centering
		\sffamily {\fontsize{18}{22}\selectfont\bfseries \Instructor\par}
		\vspace{2pt}
		{\normalsize \Role\par}
		\vspace{1.2em}
		{\large \faEnvelope\ \href{mailto:partho.dhor@bracu.ac.bd}{partho.dhor@bracu.ac.bd}\quad \textbar\quad \faEnvelope\ \href{mailto:parthosutradhor@gmail.com}{parthosutradhor@gmail.com} \par}
		\vspace{0.8em}
		\begin{minipage}{0.62\textwidth}
			\centering
			{\Large\bfseries \textcolor{YTred}{\faYoutube}\; \href{https://www.youtube.com/@ParthoSutraDhor}{@ParthoSutraDhor}}
			\par {\small (Lectures, walkthroughs, and course updates)}
		\end{minipage}%
		\hfill
		\begin{minipage}{0.30\textwidth}
			\centering
			\IfFileExists{qrcode.sty}{ \qrcode[height=3cm]{https://www.youtube.com/@ParthoSutraDhor} \par\vspace{2pt}{\scriptsize Scan for the channel} }{ \vspace{2pt}\textcolor{YTred}{\faYoutube}\par }
		\end{minipage}
	\end{tcolorbox}

	\vspace{1.6em}
	\rule{0.64\textwidth}{0.8pt}\par
	\vspace{1.2em}

	\begin{tcolorbox}
		[ colback=white, colframe=Accent, boxrule=0.8pt, arc=3pt, width=0.86\textwidth,
		left=12pt,right=12pt,top=10pt,bottom=10pt, enhanced, title=\sffamily\bfseries\large
		\faBook\ References ] \sffamily\large \ReferenceList
	\end{tcolorbox}
\end{center}

\begin{tikzpicture}[remember picture, overlay]
	\node[anchor=south west,inner sep=0pt] at ($(current page.south west)$) {%
	\begin{tikzpicture}\path[shade, left color=BannerRight, right color=BannerLeft] (0,0) rectangle (\paperwidth,0.9cm);\end{tikzpicture} };
\end{tikzpicture}
\clearpage
}

% ===============================
% Modern Box Styles (with neat tabbed titles)
% ===============================

% --- Definition (deep academic blue) ---
\definecolor{DefBack}{RGB}{228,238,250}   % soft bluish background
\definecolor{DefMain}{RGB}{0,90,190}      % deep but calm academic blue

% --- Theorem (balanced emerald green) ---
\definecolor{ThmBack}{RGB}{230,248,235}   % gentle greenish background
\definecolor{ThmMain}{RGB}{0,155,70}      % balanced rich green

% --- Problem (refined amber tone) ---
\definecolor{ProbBack}{RGB}{255,245,232}  % warm ivory-peach background
\definecolor{ProbMain}{RGB}{195,90,0}     % deep amber / rust-orange

% --- Headers (coordinated with Definition) ---
\definecolor{HeadL}{RGB}{20,55,120}       % deep navy (slightly cooler, academic tone)
\definecolor{HeadR}{RGB}{45,135,215}      % calm mid-sky blue (not too bright)

\tcbset{ parthoBase/.style={ breakable, enhanced jigsaw, boxrule=0.6pt, arc=4pt, left=10pt, right=10pt, top=10pt, bottom=10pt, coltitle=black, fonttitle=\bfseries\sffamily, fontupper=\sffamily, drop fuzzy shadow, attach boxed title to top left={yshift=-0.5mm}, % tab overlaps border neatly
boxed title style={ boxrule=0pt, arc=3pt, top=2pt, bottom=2pt, left=8pt, right=8pt, interior style={left color=tcbcolframe!10!white, right color=tcbcolframe!30!white} }, }, }

% --- Definition Box ---
\NewTColorBox{definition}{ O{Definition} }{%
parthoBase, colback=DefBack, colframe=DefMain, title={\faBook\ \ #1}, borderline west={2.5pt}{0pt}{DefMain}, overlay unbroken and first={ \node[opacity=0.06,anchor=north east,inner sep=6pt] at (frame.north east) {\scalebox{2}{\faBookOpen}}; }, }

% --- Theorem Box ---
\NewTColorBox{theorem}{ O{Theorem} }{%
parthoBase, colback=ThmBack, colframe=ThmMain, title={\faLightbulb[regular]\ \ #1}, borderline west={2.5pt}{0pt}{ThmMain}, overlay unbroken and first={ \node[opacity=0.06,anchor=north east,inner sep=6pt] at (frame.north east) {\scalebox{2}{\faLightbulb}}; }, }

% --- Problem Box ---
\NewTColorBox{problem}{ O{Problem} }{%
parthoBase, colback=ProbBack, colframe=ProbMain, title={\faQuestionCircle[regular]\ \ #1}, borderline west={2.5pt}{0pt}{ProbMain}, overlay unbroken and first={ \node[opacity=0.07,anchor=north east,inner sep=6pt] at (frame.north east) {\scalebox{2}{\faQuestionCircle}}; }, }

% --- Heading Box ---
\NewTColorBox{heading}{ O{Heading} }{%
enhanced, frame empty, interior style={left color=HeadL!95!black, right color=HeadR!90!white}, boxsep=1pt, top=10pt, bottom=10pt, left=10pt, right=10pt, coltext=white, fontupper=\sffamily\bfseries\Large }

% ==========================================
% Document
% ==========================================
\begin{document}
	\FrontPage
	\Large

    \begin{heading}
		Cartesian Product and Relation
	\end{heading}

    \begin{definition}[Relation Between Two Sets]
    A \emph{relation} $R$ from a set $A$ to a set $B$ is any subset of the Cartesian product $A \times B$, i.e.
    \[
    R \subseteq A \times B.
    \]
    If $(a,b) \in R$, we say that “$a$ is related to $b$ by $R$,” written as $a\,R\,b$ or sometimes $a \sim b$.
    \end{definition}
    
    \clearpage

    \begin{heading}
		Equivalence Relation and Examples
	\end{heading}

    \begin{definition}[Equivalence Relation]
    A relation $\sim$ on a set $A$ is called an \textbf{equivalence relation} if
\begin{enumerate}
    \item \textbf{Reflexive:} $a \sim a$, for all $a \in A$. 
    \item \textbf{Symmetric:} If $a \sim b$, then $b \sim a$,  for all $a,b \in A$.
    \item \textbf{Transitive:} If $a \sim b$ and $b \sim c$, then $a \sim c$,  for all $a,b,c \in A$.
\end{enumerate}
    \end{definition}

    \vspace{5pt}

    \begin{theorem}[Important examples of \textbf{equivalence relations}]
    
    \begin{enumerate}
        \item \textbf{Congruence modulo $n$ on $\mathbb{Z}$:}
        For a fixed integer $n>1$, define
        \[
        a \sim b \iff a \equiv b \pmod{n}.
        \]
    
        \item \textbf{Rational Difference on $\mathbb{R}$:}
        On the set of real numbers, define
        \[
        a \sim b \iff a-b \in \mathbb{Q}.
        \]
    
        \item \textbf{Congruent Triangles:}
        For triangles $\triangle ABC$ and $\triangle DEF$, define
        \[
        \triangle ABC \sim \triangle DEF \iff \triangle ABC \cong \triangle DEF
        \]
    
        \item \textbf{Parallel Lines in the Plane:}
        For lines $l_1$ and $l_2$ in $\mathbb{R}^2$, define
        \[
        l_1 \sim l_2 \iff l_1 \parallel l_2.
        \]
    
        \item \textbf{Same Slope of Nonzero Vectors:}
        In $\mathbb{R}^2\setminus\{(0,0)\}$, define
        \[
        (x_1,y_1) \sim (x_2,y_2) \iff \exists k \neq 0 \text{ such that } (x_1,y_1)=k(x_2,y_2).
        \]
    \end{enumerate}
    \end{theorem}

    \clearpage

    \begin{problem}
        \textbf{Congruence modulo $n$ on $\mathbb{Z}$:}
        For a fixed integer $n>1$, define
        \[
        a \sim b \iff a \equiv b \pmod{n}.
        \]
        Prove that $\sim$ is an equivalence relation.
    \end{problem}

    \clearpage

    \begin{problem}
        \textbf{Rational Difference on $\mathbb{R}$:}
        On the set of real numbers, define
        \[
        a \sim b \iff a-b \in \mathbb{Q}.
        \]
        Prove that $\sim$ is an equivalence relation.
    \end{problem}

    \clearpage

    \begin{problem}
        \textbf{Same Slope of Nonzero Vectors:}
        In $\mathbb{R}^2\setminus\{(0,0)\}$, define
        \[
        (x_1,y_1) \sim (x_2,y_2) \iff \exists k \neq 0 \text{ such that } (x_1,y_1)=k(x_2,y_2).
        \]
        Prove that $\sim$ is an equivalence relation.
    \end{problem}

    \clearpage

    \begin{heading}
        Equivalence Classes
    \end{heading}

    \begin{definition}[Equivalence Class]
    If $\sim$ is an equivalence relation on a set $A$ and $a \in A$,  
    the \textbf{equivalence class} of $a$ is
    \[
    [a] = \{\, x \in A \mid x \sim a \,\}.
    \]
    \end{definition}

    \vspace{10pt}

    \begin{problem}
        Define the equivalence relation $\sim$ on $\mathbb{Z}$ by
        \[
        a \sim b \iff a \equiv b \pmod{5} \iff a-b \text{ is divisible by } 5.
        \]
        Find the equivalence class of $3$.
    \end{problem}

    \clearpage

    \begin{problem}
        In $\mathbb{R}^2\setminus\{(0,0)\}$, define the equivalence relation $\sim$ by
        \[
        (x_1,y_1) \sim (x_2,y_2) \iff \exists k \neq 0 \text{ such that } (x_1,y_1)=k(x_2,y_2).
        \]
        Find the equivalence class of $(1,2)$.
    \end{problem}

    \clearpage

    \begin{theorem}
    Let $\sim$ be an equivalence relation on $A$.  
    For any $a,b \in A$, either $[a] = [b]$ or $[a] \cap [b] = \varnothing$.
    \end{theorem}

    \clearpage

    \begin{heading}
        Partition from Equivalence Relations
    \end{heading}

    \begin{theorem}[Partition Theorem]
    Every equivalence relation on $A$ partitions $A$ into disjoint equivalence classes.  
    Conversely, every partition of $A$ defines an equivalence relation.
    \end{theorem}

    \clearpage

    \begin{problem}
        Define the equivalence relation $\sim$ on $\mathbb{Z}$ by
        \[
        a \sim b \iff a \equiv b \pmod{5} \iff a-b \text{ is divisible by } 5.
        \]
        Find the partition of $\mathbb{Z}$ by $\sim$.
    \end{problem}

    \clearpage

    \begin{problem}
        In $\mathbb{R}^2\setminus\{(0,0)\}$, define the equivalence relation $\sim$ by
        \[
        (x_1,y_1) \sim (x_2,y_2) \iff \exists k \neq 0 \text{ such that } (x_1,y_1)=k(x_2,y_2).
        \]
        Give a physical interpretation of the partition of $\mathbb{R}^2\setminus\{(0,0)\}$ by $\sim$.
    \end{problem}

    \clearpage

	\begin{heading}
		Integers to Rational Numbers
	\end{heading}

	\begin{quote}
    \textit{``God made the integers; all else is the work of man.''} \\
    \hfill --- \textbf{Leopold Kronecker (1823--1891)}
    \end{quote}

    \vspace{20pt}

    \begin{definition}[Fractions as Ordered Pairs]
    Let 
    \[
    S = \mathbb{Z} \times (\mathbb{Z} \setminus \{0\})
    \]
    be the set of all ordered pairs $(a,b)$, where $a$ is any integer and $b$ is a nonzero integer.  
    We think of $(a,b)$ as the “fraction” $\dfrac{a}{b}$.
    \end{definition}

    \clearpage

    \begin{theorem}
    Define a relation $\sim$ on $S=\mathbb{Z} \times (\mathbb{Z} \setminus \{0\})$ by
    \[
    (a,b) \sim (c,d) \;\Longleftrightarrow\; ad = bc.
    \]
    Show that the relation $\sim$ is an equivalence relation on $S$.
    \end{theorem}

    \clearpage

    \begin{definition}[The Rational Numbers]
        The set of \textbf{rational numbers} is the quotient set / partition set
    \[
    \mathbb{Q} = S / \sim = \{\, [(a,b)] \mid (a,b) \in S \,\}.
    \]
    The equivalence class $[(a,b)]$ is informally written as $\dfrac{a}{b}$.
    \end{definition}

    \vfil

    \begin{problem}[Example]
        The class $[(1,2)]$ is
        \[
        [(1,2)] = \{(1,2), (2,4), (3,6), \ldots, (-1,-2), (-2,-4), \ldots\}.
        \]
        All of these pairs represent the same rational number: $\dfrac{1}{2}$.
    \end{problem}

    \vfil

    \begin{theorem}[Integers as Rationals]
        The class $[(a,1)]$ is
        \[
        [(a,1)] = \{(a,1), (2a,2), (3a,3), \ldots, (-a,-1), (-2a,-2), \ldots\}.
        \]
        All of these pairs represent the integer $a \in \mathbb{Z}$.
    \end{theorem}

    \clearpage

    \begin{theorem}[Lowest Terms Representation of Rationals]
    If $(a,b) \in S$ with $b > 0$, write $g = \gcd(a,b)$.  
    Then the class $[(a,b)]$ can be represented uniquely by
    \[
    \frac{a}{b} \;\sim\; \frac{a/g}{\,b/g\,}
    \]
    with $\gcd(a/g,\, b/g) = 1$ and denominator positive.
    \end{theorem}

    \vfill
    \vfill

    \begin{theorem}[Uniqueness of Lowest Terms]
    Every nonzero rational has a unique expression 
    \[
    \frac{m}{n}
    \quad \text{with } n > 0 \text{ and } \gcd(m,n) = 1.
    \]
    \end{theorem}

    \vfill

    \clearpage

    \begin{heading}
        Addition and Multiplication of Rational Numbers
    \end{heading}

    \begin{definition}
    For $[(a,b)], [(c,d)] \in \mathbb{Q}$, define:
    \[
    [(a,b)] + [(c,d)] := [(ad + bc,\, bd)], \qquad
    [(a,b)] \cdot [(c,d)] := [(ac,\, bd)].
    \]
    \end{definition}

    \vspace{10pt}

    \begin{theorem}[Well-definedness]
    These operations do not depend on the representatives chosen.
    \end{theorem}

    \clearpage

    \begin{heading}
        Rationals as a Field
    \end{heading}

    \clearpage

    \begin{heading}
        Order on the Rationals
    \end{heading}

    \begin{definition}
        For $[(a,b)], [(c,d)] \in \mathbb{Q}$ with $b,d > 0$, define
        \[
        [(a,b)] < [(c,d)] \;\Longleftrightarrow\; ad < bc.\]
    \end{definition}

    \clearpage

    \begin{heading}
        Density of Rationals
    \end{heading}

    \begin{theorem}[Density]
    Between any two rationals $x < y$, there exists another rational $z$ with $x < z < y$.
    \end{theorem}

	\EndPage
\end{document}