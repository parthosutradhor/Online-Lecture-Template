% ===============================
% Worksheet / Manual Template
% ===============================
\documentclass[12pt, a4paper]{article}

% ===============================
% Metadata
% ===============================
\newcommand*{\CourseCode}{MAT221}
\newcommand*{\CourseTitle}{Complex Variables \& Laplace Transform}
\newcommand*{\Chapter}{Complex Functions}
\newcommand*{\Topics}{Exponential, Logarithmic, Circular and Hyperbolic Functions}
\newcommand*{\DocDate}{October 22, 2025}


\newcommand{\ReferenceList}{%
\begin{enumerate}[leftmargin=2em, label={[\arabic{*}]}]
	\item Stephen Abbott, \textit{Understanding Analysis}, 2nd Edition, Springer, 2015.

	\item Terence Tao, \textit{Analysis I}, 3rd Edition, Texts and Readings in Mathematics, Hindustan Book Agency, 2016.
\end{enumerate}
}

\newcommand*{\University}{BRAC University}
\newcommand*{\Department}{Department of Mathematics and Natural Sciences}
\newcommand*{\LogoPath}{Brac_University_Logo} % logo filename (no extension)
\newcommand*{\Instructor}{Partho Sutra Dhor}
\newcommand*{\Role}{Lecturer, BRAC University, Dhaka-1212}

% ---------- Encoding & Fonts ----------
\usepackage[T1]{fontenc}
\usepackage[utf8]{inputenc}
\usepackage{lmodern}
\usepackage{microtype}
\IfFileExists{qrcode.sty}{\usepackage{qrcode}}{}
\usepackage{enumitem}

% ---------- Icons & Colors ----------
\usepackage{fontawesome5} % icons
\usepackage[dvipsnames, table]{xcolor}
\definecolor{YTred}{RGB}{255,0,0}
\definecolor{BannerLeft}{RGB}{30,64,175}
\definecolor{BannerRight}{RGB}{56,189,248}
\definecolor{Accent}{RGB}{20,120,200}

% ---------- Page Setup ----------
\usepackage[top=0.6in, bottom=1.0in, left=0.7in, right=0.7in]{geometry}
\usepackage{setspace}
\setstretch{1.2}
\setlength{\parindent}{0pt}
\setlength{\parskip}{6pt}
\pagestyle{empty}

% ---------- Math & Graphics ----------
\usepackage{amsmath, amssymb, amsfonts, mathtools, bm}
\usepackage{graphicx}
\usepackage{tikz, pgfplots}
\pgfplotsset{compat=1.18}
\usetikzlibrary{calc}

% ---------- Tables & Lists ----------
\usepackage{array, booktabs, multirow, colortbl, float}
\renewcommand{\arraystretch}{1.25}
\usepackage{enumitem}
\setlist{leftmargin=*,itemsep=2pt}

% ---------- Boxes & Links ----------
\usepackage[most]{tcolorbox}
\tcbuselibrary{breakable,skins}
\usepackage[
	colorlinks=true,
	linkcolor=MidnightBlue,
	citecolor=OliveGreen,
	urlcolor=RoyalBlue
]{hyperref}
\usepackage[nameinlink, noabbrev]{cleveref}

\DeclareMathOperator{\card}{card}
\newcommand{\N}{\mathbb{N}}
\newcommand{\R}{\mathbb{R}}
\newcommand{\Q}{\mathbb{Q}}
\newcommand{\Z}{\mathbb{Z}}
\newcommand{\C}{\mathbb{C}}

% ===============================
% Cover Page (styled with icons)
% ===============================
\newcommand{\FrontPage}{%
\begin{titlepage}
	\centering

	% --- Gradient banner across the top ---
	\begin{tikzpicture}[remember picture, overlay]
		\node[anchor=north west,inner sep=0pt] at ($(current page.north west)$) {%
		\begin{tikzpicture}\path[shade, left color=BannerLeft, right color=BannerRight] (0,0) rectangle (\paperwidth,1.8cm);\end{tikzpicture} };
	\end{tikzpicture}
	\vspace*{1.0cm}

	% --- Logo & University Name ---
	\includegraphics[width=0.18\textwidth]{\LogoPath}
	\par
	\vspace{6pt}
	{\Large\sffamily\bfseries \University\par} {\large\sffamily \Department\par}

	% --- Title block ---
	\vspace{24pt}
	{\normalsize\sffamily\scshape Lecture on\par}
	\vspace{6pt}
	{\Large\sffamily\bfseries \CourseTitle\ (\CourseCode)\par}

	\vspace{18pt}
	{\Huge\sffamily\bfseries \textcolor{Accent}{\Chapter}\par}
	\vspace{10pt}
	{\Large\sffamily\bfseries \textcolor{teal}{\Topics}\par}
	\vfill
	\vspace{14pt}
	\rule{0.55\textwidth}{0.8pt}\par
	\vspace{10pt}
	\vfill
	{\large\sffamily\scshape \DocDate\par}
	\vfill
	\vspace{10pt}
	{\large\sffamily\scshape Conducted by\par}
	{\LARGE\sffamily\bfseries \Instructor\par} {\large\sffamily \Role\par}

	% --- Contact / Links with icons ---
	\vspace{18pt}
	\begin{tcolorbox}
		[ colback=gray!3!white, colframe=gray!40!black, boxrule=0.6pt, arc=2pt, left=8pt,right=8pt,top=8pt,bottom=8pt,
		width=0.82\textwidth, enhanced ]
		\centering
		\sffamily {\large \faEnvelope\;\href{mailto:partho.dhor@bracu.ac.bd}{partho.dhor@bracu.ac.bd} \quad\textbar\quad \faEnvelope\;\href{mailto:parthosutradhor@gmail.com}{parthosutradhor@gmail.com} \\[4pt] For updates subscribe on \quad \textcolor{YTred}{\faYoutube}\; \href{https://www.youtube.com/@ParthoSutraDhor}{@ParthoSutraDhor} }
	\end{tcolorbox}

	\vspace{10pt}
	\begin{tikzpicture}[remember picture, overlay]
		\node[anchor=south west,inner sep=0pt] at ($(current page.south west)$) {%
		\begin{tikzpicture}\path[shade, left color=BannerRight, right color=BannerLeft] (0,0) rectangle (\paperwidth,0.9cm);\end{tikzpicture} };
	\end{tikzpicture}
\end{titlepage}%
}

% ===============================
% End Page
% ===============================
\newcommand{\EndPage}{%
\clearpage
\thispagestyle{empty}
\begin{tikzpicture}[remember picture, overlay]
	\node[anchor=north west,inner sep=0pt] at ($(current page.north west)$) {%
	\begin{tikzpicture}\path[shade, left color=BannerLeft, right color=BannerRight] (0,0) rectangle (\paperwidth,1.2cm);\end{tikzpicture} };
\end{tikzpicture}

\vspace*{0.8cm}
\begin{center}
	{\fontsize{38}{42}\selectfont\sffamily\bfseries Thank You!\par}
	\vspace{0.4em}
	{\large\sffamily We’d love your questions and feedback.\par}
	\vspace{1.2em}
	\rule{0.5\textwidth}{0.8pt}\par
	\vspace{1.2em}

	\begin{tcolorbox}
		[ colback=gray!2!white, colframe=gray!35!black, boxrule=0.7pt, arc=3pt,
		width=0.86\textwidth, left=10pt,right=10pt,top=10pt,bottom=10pt, enhanced ]
		\centering
		\sffamily {\fontsize{18}{22}\selectfont\bfseries \Instructor\par}
		\vspace{2pt}
		{\normalsize \Role\par}
		\vspace{1.2em}
		{\large \faEnvelope\ \href{mailto:partho.dhor@bracu.ac.bd}{partho.dhor@bracu.ac.bd}\quad \textbar\quad \faEnvelope\ \href{mailto:parthosutradhor@gmail.com}{parthosutradhor@gmail.com} \par}
		\vspace{0.8em}
		\begin{minipage}{0.62\textwidth}
			\centering
			{\Large\bfseries \textcolor{YTred}{\faYoutube}\; \href{https://www.youtube.com/@ParthoSutraDhor}{@ParthoSutraDhor}}
			\par {\small (Lectures, walkthroughs, and course updates)}
		\end{minipage}%
		\hfill
		\begin{minipage}{0.30\textwidth}
			\centering
			\IfFileExists{qrcode.sty}{ \qrcode[height=3cm]{https://www.youtube.com/@ParthoSutraDhor} \par\vspace{2pt}{\scriptsize Scan for the channel} }{ \vspace{2pt}\textcolor{YTred}{\faYoutube}\par }
		\end{minipage}
	\end{tcolorbox}

	\vspace{1.6em}
	\rule{0.64\textwidth}{0.8pt}\par
	\vspace{1.2em}

	\begin{tcolorbox}
		[ colback=white, colframe=Accent, boxrule=0.8pt, arc=3pt, width=0.86\textwidth,
		left=12pt,right=12pt,top=10pt,bottom=10pt, enhanced, title=\sffamily\bfseries\large
		\faBook\ References ] \sffamily\large \ReferenceList
	\end{tcolorbox}
\end{center}

\begin{tikzpicture}[remember picture, overlay]
	\node[anchor=south west,inner sep=0pt] at ($(current page.south west)$) {%
	\begin{tikzpicture}\path[shade, left color=BannerRight, right color=BannerLeft] (0,0) rectangle (\paperwidth,0.9cm);\end{tikzpicture} };
\end{tikzpicture}
\clearpage
}

% ===============================
% Modern Box Styles (with neat tabbed titles)
% ===============================

% --- Definition (deep academic blue) ---
\definecolor{DefBack}{RGB}{228,238,250}   % soft bluish background
\definecolor{DefMain}{RGB}{0,90,190}      % deep but calm academic blue

% --- Theorem (balanced emerald green) ---
\definecolor{ThmBack}{RGB}{230,248,235}   % gentle greenish background
\definecolor{ThmMain}{RGB}{0,155,70}      % balanced rich green

% --- Problem (refined amber tone) ---
\definecolor{ProbBack}{RGB}{255,245,232}  % warm ivory-peach background
\definecolor{ProbMain}{RGB}{195,90,0}     % deep amber / rust-orange

% --- Headers (coordinated with Definition) ---
\definecolor{HeadL}{RGB}{20,55,120}       % deep navy (slightly cooler, academic tone)
\definecolor{HeadR}{RGB}{45,135,215}      % calm mid-sky blue (not too bright)

\tcbset{ parthoBase/.style={ breakable, enhanced jigsaw, boxrule=0.6pt, arc=4pt, left=10pt, right=10pt, top=10pt, bottom=10pt, coltitle=black, fonttitle=\bfseries\sffamily, fontupper=\sffamily, drop fuzzy shadow, attach boxed title to top left={yshift=-0.5mm}, % tab overlaps border neatly
boxed title style={ boxrule=0pt, arc=3pt, top=2pt, bottom=2pt, left=8pt, right=8pt, interior style={left color=tcbcolframe!10!white, right color=tcbcolframe!30!white} }, }, }

% --- Definition Box ---
\NewTColorBox{definition}{ O{Definition} }{%
parthoBase, colback=DefBack, colframe=DefMain, title={\faBook\ \ #1}, borderline west={2.5pt}{0pt}{DefMain}, overlay unbroken and first={ \node[opacity=0.06,anchor=north east,inner sep=6pt] at (frame.north east) {\scalebox{2}{\faBookOpen}}; }, }

% --- Theorem Box ---
\NewTColorBox{theorem}{ O{Theorem} }{%
parthoBase, colback=ThmBack, colframe=ThmMain, title={\faLightbulb[regular]\ \ #1}, borderline west={2.5pt}{0pt}{ThmMain}, overlay unbroken and first={ \node[opacity=0.06,anchor=north east,inner sep=6pt] at (frame.north east) {\scalebox{2}{\faLightbulb}}; }, }

% --- Problem Box ---
\NewTColorBox{problem}{ O{Problem} }{%
parthoBase, colback=ProbBack, colframe=ProbMain, title={\faQuestionCircle[regular]\ \ #1}, borderline west={2.5pt}{0pt}{ProbMain}, overlay unbroken and first={ \node[opacity=0.07,anchor=north east,inner sep=6pt] at (frame.north east) {\scalebox{2}{\faQuestionCircle}}; }, }

% --- Heading Box ---
\NewTColorBox{heading}{ O{Heading} }{%
enhanced, frame empty, interior style={left color=HeadL!95!black, right color=HeadR!90!white}, boxsep=1pt, top=10pt, bottom=10pt, left=10pt, right=10pt, coltext=white, fontupper=\sffamily\bfseries\Large }


% ==========================================
% Document
% ==========================================
\begin{document}
\FrontPage
\Large

\begin{heading}
	Single valued and multi valued functions
\end{heading}

\begin{definition}[Single-valued Function]
	A function \( f: D \to \C \) is called single-valued if for every input \( z \in D \), there is exactly one output \( w \in \C \) such that \( f(z) = w \).
\end{definition}

\vfill

\begin{definition}[Multi-valued Function]
	A function \( f: D \to \mathcal{P}(\C) \) is called multi-valued if for some input \( z \in D \), there are multiple outputs \( w_1, w_2, \ldots, w_n \in \C \) such that \( f(z) = \{w_1, w_2, \ldots, w_n\} \) with \( n > 1 \).\\[8pt]

	A multiple-valued function can be considered as a collection of single-valued functions, each member of which is called a branch of the function.
\end{definition}

\vfill

\clearpage

\begin{heading}
	Why some functions are multi-valued?
\end{heading}

\clearpage

\begin{heading}
	The Exponential Function
\end{heading}

\clearpage

\begin{theorem}[Complex Exponential Function is Periodic]
	The complex exponential function \( e^z \) is periodic with period \( 2\pi i \).
\end{theorem}

\clearpage

\begin{problem}
	Find all values of \( z \in \C \) such that \[e^z = -1 + i\sqrt{3}.\]
\end{problem}

\clearpage

\begin{problem}
	Find all values of \( z \in \C \) such that \[e^{4z} = -1\]
\end{problem}

\clearpage

\begin{heading}
	The Logarithmic Function
\end{heading}

\clearpage

\begin{definition}[Complex Logarithm]
	The complex logarithm of a non-zero complex number \( z \) is defined as \[\log z = \ln|z| + i\arg z\] where \( \arg z \) is the argument of \( z \).\\[15pt]
	Note: The complex logarithm is a multi-valued function due to the multi-valued nature of the argument \( \arg z \).
\end{definition}

\clearpage

\begin{problem}
Show that:

\begin{enumerate}[label=(\roman*)]
    \item \(\ln(-1 + \sqrt{3}i) = \ln 2 + 2\left(n + \tfrac{1}{3}\right)\pi i\)
    
    \item \(\ln(1 - i) = \tfrac{1}{2}\ln 2 + \left(2n + \tfrac{7}{4}\right)\pi i\)
    
    \item \(\ln(i^{1/2}) = \left(n + \tfrac{1}{4}\right)\pi i\)
\end{enumerate}
\end{problem}

\clearpage

\begin{heading}
	Circular Functions
\end{heading}

\vspace{10pt}

\begin{definition}[Complex Sine and Cosine Functions]
	The complex sine and cosine functions are defined as follows:
	\[
	\sin z = \frac{e^{iz} - e^{-iz}}{2i}, \quad \cos z = \frac{e^{iz} + e^{-iz}}{2}
	\]
	for all \( z \in \C \).
\end{definition}

\clearpage

\begin{heading}
	Inverse Circular Functions
\end{heading}

\vspace{10pt}

\begin{definition}[Inverse Circular Functions]
	The inverse circular functions for complex numbers are defined as follows:
	\begin{align*}
	\sin^{-1} z &= -i \ln\left(iz + \sqrt{1 - z^2}\right) \\[6pt]
	\cos^{-1} z &= -i \ln\left(z + \sqrt{z^2 - 1}\right) \\[6pt]
	\tan^{-1} z &= \frac{i}{2} \ln\left(\frac{i + z}{i - z}\right) \\[6pt]
	\cot^{-1} z &= \frac{i}{2} \ln\left(\frac{z - i}{z + i}\right) \\[6pt]
	\sec^{-1} z &= -i \ln\left(\frac{1 + \sqrt{1 - z^2}}{z}\right) \\[6pt]
	\cosec^{-1} z &= -i \ln\left(\frac{i + \sqrt{z^2 - 1}}{z}\right)
	\end{align*}
	for all \( z \in \C \).
\end{definition}

\clearpage

\begin{problem}
	Prove that for all \( z \in \C \):
	\[\sin^{-1} z = -i \ln\left(iz + \sqrt{1 - z^2}\right)\]
\end{problem}

\clearpage

\begin{problem}
	Prove that for all \( z \in \C \):
	\[\sec^{-1} z = -i \ln\left(\frac{1 + \sqrt{1 - z^2}}{z}\right)\]
\end{problem}

\clearpage

\begin{problem}
	Prove that for all \( z \in \C \):
	\[\cot^{-1} z = \frac{i}{2} \ln\left(\frac{z - i}{z + i}\right)\]
\end{problem}

\clearpage

\begin{problem}
	Solve for \( z \in \C \):
	\[\sin z = 2\]
\end{problem}

\clearpage

\begin{heading}
	Hyperbolic Functions
\end{heading}

\vspace{10pt}

\begin{definition}[Sine Hyperbolic and Cosine Hyperbolic Functions]
	The complex sine and cosine functions are defined as follows:
	\[
	\sinh z = \frac{e^{z} - e^{-z}}{2}, \quad \cosh z = \frac{e^{z} + e^{-z}}{2}
	\]	
	for all \( z \in \C \).
\end{definition}

\clearpage

\begin{heading}
	Inverse Hyperbolic Functions
\end{heading}

\vspace{10pt}

\begin{definition}[Inverse Hyperbolic Functions]
The inverse hyperbolic functions for complex numbers are defined as follows:
\[
\begin{aligned}
\sinh^{-1} z &= \ln\left(z + \sqrt{z^2 + 1}\right) \\[6pt]
\cosh^{-1} z &= \ln\left(z + \sqrt{z^2 - 1}\right) \\[6pt]
\tanh^{-1} z &= \frac{1}{2} \ln\left(\frac{1 + z}{1 - z}\right) \\[6pt]
\coth^{-1} z &= \frac{1}{2} \ln\left(\frac{z + 1}{z - 1}\right) \\[6pt]
\sech^{-1} z &= \ln\left(\frac{1 + \sqrt{1 - z^2}}{z}\right)\\[6pt]
\cosech^{-1} z &= \ln\left(\frac{1 + \sqrt{1 + z^2}}{z}\right)
\end{aligned}
\]
for all \( z \in \C \).
\end{definition}

\clearpage

\begin{problem}
	Prove that for all \( z \in \C \):
	\[\cosh^{-1} z = \log\left(z + \sqrt{z^2 - 1}\right)\]
\end{problem}

\clearpage

\begin{problem}
	Prove that for all \( z \in \C \):
	\[\cosech^{-1} z = \ln\left(\frac{1 + \sqrt{1 + z^2}}{z}\right)\]
\end{problem}

\clearpage

\begin{problem}
	Prove that for all \( z \in \C \):
	\[\tanh^{-1} z = \frac{1}{2} \log\left(\frac{1 + z}{1 - z}\right)\]
\end{problem}

\clearpage

\begin{problem}
	Solve for \( z \in \C \):
	\[\cosh z = 3i\]
\end{problem}

\clearpage

\begin{heading}
	Properties of Circular and Hyperbolic Functions
\end{heading}

\vspace{10pt}

\begin{theorem}
For all \( z \in \C \):
\[\sin z = \sin x \cosh y + i \cos x \sinh y\]
\[\cos z = \cos x \cosh y - i \sin x \sinh y\]
where \( z = x + iy \), \( x, y \in \R \).
\end{theorem}

\vspace{10pt}

\begin{theorem}
For all \( z \in \C \):
\[\sinh z = \sinh x \cos y + i \cosh x \sin y\]
\[\cosh z = \cosh x \cos y + i \sinh x \sin y\]
where \( z = x + iy \), \( x, y \in \R \).
\end{theorem}

\vspace{10pt}

\begin{theorem}
For all \( z \in \C \):
\begin{align*}
\sin(iz) &= i\sinh z, \\[4pt]
\cos(iz) &= \cosh z, \\[4pt]
\sinh(iz) &= i\sin z, \\[4pt]
\cosh(iz) &= \cos z.
\end{align*}
\end{theorem}

\clearpage

\begin{theorem}
For all \( z \in \C \):
\[\\cos^2 z + \sin^2 z = 1\]
\[cosh^2 z - sinh^2 z = 1\]
\end{theorem}

\vspace{10pt}

\begin{theorem}
For all \( z \in \C \):
\[\frac{d}{dz}(\sin z) = \cos z, \quad \frac{d}{dz}(\cos z) = -\sin z\]
\[\frac{d}{dz}(\sinh z) = \cosh z, \quad \frac{d}{dz}(\cosh z) = \sinh z\]
\end{theorem}



\EndPage
\end{document}