% ===============================
% Worksheet / Manual Template
% ===============================
\documentclass[12pt, a4paper]{article}

% ===============================
% Metadata
% ===============================
\newcommand*{\CourseCode}{MAT215}
\newcommand*{\CourseTitle}{Complex Variables \& Laplace Transform}
\newcommand*{\Chapter}{Limit and Continuity of Complex Functions}
\newcommand*{\Topics}{L'Hospital's Rule and Existence of Limits}
\newcommand*{\DocDate}{October 27, 2025}


\newcommand{\ReferenceList}{%
\begin{enumerate}[leftmargin=2em, label={[\arabic{*}]}]
	\item Stephen Abbott, \textit{Understanding Analysis}, 2nd Edition, Springer, 2015.

	\item Terence Tao, \textit{Analysis I}, 3rd Edition, Texts and Readings in Mathematics, Hindustan Book Agency, 2016.
\end{enumerate}
}

\newcommand*{\University}{BRAC University}
\newcommand*{\Department}{Department of Mathematics and Natural Sciences}
\newcommand*{\LogoPath}{Brac_University_Logo} % logo filename (no extension)
\newcommand*{\Instructor}{Partho Sutra Dhor}
\newcommand*{\Role}{Lecturer, BRAC University, Dhaka-1212}

% ---------- Encoding & Fonts ----------
\usepackage[T1]{fontenc}
\usepackage[utf8]{inputenc}
\usepackage{lmodern}
\usepackage{microtype}
\IfFileExists{qrcode.sty}{\usepackage{qrcode}}{}
\usepackage{enumitem}

% ---------- Icons & Colors ----------
\usepackage{fontawesome5} % icons
\usepackage[dvipsnames, table]{xcolor}
\definecolor{YTred}{RGB}{255,0,0}
\definecolor{BannerLeft}{RGB}{30,64,175}
\definecolor{BannerRight}{RGB}{56,189,248}
\definecolor{Accent}{RGB}{20,120,200}

% ---------- Page Setup ----------
\usepackage[top=0.6in, bottom=1.0in, left=0.7in, right=0.7in]{geometry}
\usepackage{setspace}
\setstretch{1.2}
\setlength{\parindent}{0pt}
\setlength{\parskip}{6pt}
\pagestyle{empty}

% ---------- Math & Graphics ----------
\usepackage{amsmath, amssymb, amsfonts, mathtools, bm}
\usepackage{graphicx}
\usepackage{tikz, pgfplots}
\pgfplotsset{compat=1.18}
\usetikzlibrary{calc}

% ---------- Tables & Lists ----------
\usepackage{array, booktabs, multirow, colortbl, float}
\renewcommand{\arraystretch}{1.25}
\usepackage{enumitem}
\setlist{leftmargin=*,itemsep=2pt}

% ---------- Boxes & Links ----------
\usepackage[most]{tcolorbox}
\tcbuselibrary{breakable,skins}
\usepackage[
	colorlinks=true,
	linkcolor=MidnightBlue,
	citecolor=OliveGreen,
	urlcolor=RoyalBlue
]{hyperref}
\usepackage[nameinlink, noabbrev]{cleveref}

\DeclareMathOperator{\card}{card}
\newcommand{\N}{\mathbb{N}}
\newcommand{\R}{\mathbb{R}}
\newcommand{\Q}{\mathbb{Q}}
\newcommand{\Z}{\mathbb{Z}}
\newcommand{\C}{\mathbb{C}}

% ===============================
% Cover Page (styled with icons)
% ===============================
\newcommand{\FrontPage}{%
\begin{titlepage}
	\centering

	% --- Gradient banner across the top ---
	\begin{tikzpicture}[remember picture, overlay]
		\node[anchor=north west,inner sep=0pt] at ($(current page.north west)$) {%
		\begin{tikzpicture}\path[shade, left color=BannerLeft, right color=BannerRight] (0,0) rectangle (\paperwidth,1.8cm);\end{tikzpicture} };
	\end{tikzpicture}
	\vspace*{1.0cm}

	% --- Logo & University Name ---
	\includegraphics[width=0.18\textwidth]{\LogoPath}
	\par
	\vspace{6pt}
	{\Large\sffamily\bfseries \University\par} {\large\sffamily \Department\par}

	% --- Title block ---
	\vspace{24pt}
	{\normalsize\sffamily\scshape Lecture on\par}
	\vspace{6pt}
	{\Large\sffamily\bfseries \CourseTitle\ (\CourseCode)\par}

	\vspace{18pt}
	{\Huge\sffamily\bfseries \textcolor{Accent}{\Chapter}\par}
	\vspace{10pt}
	{\Large\sffamily\bfseries \textcolor{teal}{\Topics}\par}
	\vfill
	\vspace{14pt}
	\rule{0.55\textwidth}{0.8pt}\par
	\vspace{10pt}
	\vfill
	{\large\sffamily\scshape \DocDate\par}
	\vfill
	\vspace{10pt}
	{\large\sffamily\scshape Conducted by\par}
	{\LARGE\sffamily\bfseries \Instructor\par} {\large\sffamily \Role\par}

	% --- Contact / Links with icons ---
	\vspace{18pt}
	\begin{tcolorbox}
		[ colback=gray!3!white, colframe=gray!40!black, boxrule=0.6pt, arc=2pt, left=8pt,right=8pt,top=8pt,bottom=8pt,
		width=0.82\textwidth, enhanced ]
		\centering
		\sffamily {\large \faEnvelope\;\href{mailto:partho.dhor@bracu.ac.bd}{partho.dhor@bracu.ac.bd} \quad\textbar\quad \faEnvelope\;\href{mailto:parthosutradhor@gmail.com}{parthosutradhor@gmail.com} \\[4pt] For updates subscribe on \quad \textcolor{YTred}{\faYoutube}\; \href{https://www.youtube.com/@ParthoSutraDhor}{@ParthoSutraDhor} }
	\end{tcolorbox}

	\vspace{10pt}
	\begin{tikzpicture}[remember picture, overlay]
		\node[anchor=south west,inner sep=0pt] at ($(current page.south west)$) {%
		\begin{tikzpicture}\path[shade, left color=BannerRight, right color=BannerLeft] (0,0) rectangle (\paperwidth,0.9cm);\end{tikzpicture} };
	\end{tikzpicture}
\end{titlepage}%
}

% ===============================
% End Page
% ===============================
\newcommand{\EndPage}{%
\clearpage
\thispagestyle{empty}
\begin{tikzpicture}[remember picture, overlay]
	\node[anchor=north west,inner sep=0pt] at ($(current page.north west)$) {%
	\begin{tikzpicture}\path[shade, left color=BannerLeft, right color=BannerRight] (0,0) rectangle (\paperwidth,1.2cm);\end{tikzpicture} };
\end{tikzpicture}

\vspace*{0.8cm}
\begin{center}
	{\fontsize{38}{42}\selectfont\sffamily\bfseries Thank You!\par}
	\vspace{0.4em}
	{\large\sffamily We’d love your questions and feedback.\par}
	\vspace{1.2em}
	\rule{0.5\textwidth}{0.8pt}\par
	\vspace{1.2em}

	\begin{tcolorbox}
		[ colback=gray!2!white, colframe=gray!35!black, boxrule=0.7pt, arc=3pt,
		width=0.86\textwidth, left=10pt,right=10pt,top=10pt,bottom=10pt, enhanced ]
		\centering
		\sffamily {\fontsize{18}{22}\selectfont\bfseries \Instructor\par}
		\vspace{2pt}
		{\normalsize \Role\par}
		\vspace{1.2em}
		{\large \faEnvelope\ \href{mailto:partho.dhor@bracu.ac.bd}{partho.dhor@bracu.ac.bd}\quad \textbar\quad \faEnvelope\ \href{mailto:parthosutradhor@gmail.com}{parthosutradhor@gmail.com} \par}
		\vspace{0.8em}
		\begin{minipage}{0.62\textwidth}
			\centering
			{\Large\bfseries \textcolor{YTred}{\faYoutube}\; \href{https://www.youtube.com/@ParthoSutraDhor}{@ParthoSutraDhor}}
			\par {\small (Lectures, walkthroughs, and course updates)}
		\end{minipage}%
		\hfill
		\begin{minipage}{0.30\textwidth}
			\centering
			\IfFileExists{qrcode.sty}{ \qrcode[height=3cm]{https://www.youtube.com/@ParthoSutraDhor} \par\vspace{2pt}{\scriptsize Scan for the channel} }{ \vspace{2pt}\textcolor{YTred}{\faYoutube}\par }
		\end{minipage}
	\end{tcolorbox}

	\vspace{1.6em}
	\rule{0.64\textwidth}{0.8pt}\par
	\vspace{1.2em}

	\begin{tcolorbox}
		[ colback=white, colframe=Accent, boxrule=0.8pt, arc=3pt, width=0.86\textwidth,
		left=12pt,right=12pt,top=10pt,bottom=10pt, enhanced, title=\sffamily\bfseries\large
		\faBook\ References ] \sffamily\large \ReferenceList
	\end{tcolorbox}
\end{center}

\begin{tikzpicture}[remember picture, overlay]
	\node[anchor=south west,inner sep=0pt] at ($(current page.south west)$) {%
	\begin{tikzpicture}\path[shade, left color=BannerRight, right color=BannerLeft] (0,0) rectangle (\paperwidth,0.9cm);\end{tikzpicture} };
\end{tikzpicture}
\clearpage
}

% ===============================
% Modern Box Styles (with neat tabbed titles)
% ===============================

% --- Definition (deep academic blue) ---
\definecolor{DefBack}{RGB}{228,238,250}   % soft bluish background
\definecolor{DefMain}{RGB}{0,90,190}      % deep but calm academic blue

% --- Theorem (balanced emerald green) ---
\definecolor{ThmBack}{RGB}{230,248,235}   % gentle greenish background
\definecolor{ThmMain}{RGB}{0,155,70}      % balanced rich green

% --- Problem (refined amber tone) ---
\definecolor{ProbBack}{RGB}{255,245,232}  % warm ivory-peach background
\definecolor{ProbMain}{RGB}{195,90,0}     % deep amber / rust-orange

% --- Headers (coordinated with Definition) ---
\definecolor{HeadL}{RGB}{20,55,120}       % deep navy (slightly cooler, academic tone)
\definecolor{HeadR}{RGB}{45,135,215}      % calm mid-sky blue (not too bright)

\tcbset{ parthoBase/.style={ breakable, enhanced jigsaw, boxrule=0.6pt, arc=4pt, left=10pt, right=10pt, top=10pt, bottom=10pt, coltitle=black, fonttitle=\bfseries\sffamily, fontupper=\sffamily, drop fuzzy shadow, attach boxed title to top left={yshift=-0.5mm}, % tab overlaps border neatly
boxed title style={ boxrule=0pt, arc=3pt, top=2pt, bottom=2pt, left=8pt, right=8pt, interior style={left color=tcbcolframe!10!white, right color=tcbcolframe!30!white} }, }, }

% --- Definition Box ---
\NewTColorBox{definition}{ O{Definition} }{%
parthoBase, colback=DefBack, colframe=DefMain, title={\faBook\ \ #1}, borderline west={2.5pt}{0pt}{DefMain}, overlay unbroken and first={ \node[opacity=0.06,anchor=north east,inner sep=6pt] at (frame.north east) {\scalebox{2}{\faBookOpen}}; }, }

% --- Theorem Box ---
\NewTColorBox{theorem}{ O{Theorem} }{%
parthoBase, colback=ThmBack, colframe=ThmMain, title={\faLightbulb[regular]\ \ #1}, borderline west={2.5pt}{0pt}{ThmMain}, overlay unbroken and first={ \node[opacity=0.06,anchor=north east,inner sep=6pt] at (frame.north east) {\scalebox{2}{\faLightbulb}}; }, }

% --- Problem Box ---
\NewTColorBox{problem}{ O{Problem} }{%
parthoBase, colback=ProbBack, colframe=ProbMain, title={\faQuestionCircle[regular]\ \ #1}, borderline west={2.5pt}{0pt}{ProbMain}, overlay unbroken and first={ \node[opacity=0.07,anchor=north east,inner sep=6pt] at (frame.north east) {\scalebox{2}{\faQuestionCircle}}; }, }

% --- Heading Box ---
\NewTColorBox{heading}{ O{Heading} }{%
enhanced, frame empty, interior style={left color=HeadL!95!black, right color=HeadR!90!white}, boxsep=1pt, top=10pt, bottom=10pt, left=10pt, right=10pt, coltext=white, fontupper=\sffamily\bfseries\Large }


% ==========================================
% Document
% ==========================================
\begin{document}
\FrontPage
\Large

\begin{heading}
	Undefined vs Indeterminate Forms
\end{heading}

\clearpage

\begin{heading}
	An Intuitive Approach to Limits
\end{heading}

\vspace{10pt}

\begin{problem}
	Find the limit, if it exists:
	\[ \lim_{x \to 2} x^2 \]
\end{problem}

\vfill

\begin{problem}
	Find the limit, if it exists:
	\[ \lim_{x \to 2} \frac{x^2-2}{x-2} \]
\end{problem}

\vfill

\clearpage

\begin{problem}
	Find the limit, if it exists:
	\[ \lim_{x \to 0} \left(1+x\right)^{\frac{1}{x}} \]
\end{problem}

\clearpage

\begin{heading}
	Left and Right Hand Limits
\end{heading}

\vspace{10pt}

\begin{problem}
	Find the limit, if it exists:
	\[ \lim_{x \to 5} \frac{x}{5-x} \]
\end{problem}

\clearpage

\begin{heading}
	Limit of a Real-valued Function
\end{heading}

\vspace{10pt}

\begin{definition}[Limit of a Real-valued Function]
	Let \( f: D \to \R \) be a function defined on a domain \( D \subseteq \R \). We say that the limit of \( f(x) \) as \( x \) approaches \( c \) is \( L \), denoted by
	\[
		\lim_{x \to c} f(x) = L,
	\]
	if for every \( \epsilon > 0 \), there exists a \( \delta > 0 \) such that for all \( x \in D \) with \( 0 < |x - c| < \delta \), it follows that \( |f(x) - L| < \epsilon \).
\end{definition}

\clearpage

\begin{heading}
	Limit of a Complex-valued Function
\end{heading}

\vspace{10pt}

\begin{definition}[Limit of a Complex-valued Function]
	Let \( f: D \to \C \) be a function defined on a domain \( D \subseteq \C \). We say that the limit of \( f(z) \) as \( z \) approaches \( z_0 \) is \( L \), denoted by
	\[
		\lim_{z \to z_0} f(z) = L,
	\]
	if for every \( \epsilon > 0 \), there exists a \( \delta > 0 \) such that for all \( z \in D \) with \( 0 < |z - z_0| < \delta \), it follows that \( |f(z) - L| < \epsilon \).
\end{definition}

\clearpage

\begin{heading}
	How many ways can z approach a point \( z_0 \)?
\end{heading}

\vspace{10pt}

\begin{definition}[Limit (path approach)]
	If in any path that \( z \) approaches \( z_0 \), the limit of \( f(z) \) is the same value \( L \), then we say that the limit of \( f(z) \) as \( z \) approaches \( z_0 \) exists and is equal to \( L \). Otherwise, we say that the limit does not exist.
\end{definition}

\clearpage

\begin{problem}
	Show that the limit
	\[ \lim_{z \to 0} \frac{\bar{z}}{z} \]
	does not exist.
\end{problem}

\clearpage

\begin{problem}
	Show that the limit
	\[ \lim_{z \to a} \frac{\bar{z}}{z} \]
	does not exist for any \( a \in \C\).
\end{problem}

\clearpage

\begin{problem}
	Show that the limit
	\[ \lim_{z \to 0} \frac{xy}{|z|^2} \]
	does not exist where \( z = x + iy \).
\end{problem}

\clearpage

\begin{heading}
	Properties of Limits
\end{heading}
\vspace{10pt}

\begin{theorem}
	Let \( f(z) \) and \( g(z) \) be functions defined on a domain \( D \subseteq \C \), and let \( z_0 \) be a limit point of \( D \). If
	\[
		\lim_{z \to z_0} f(z) = L_1 \quad \text{and} \quad \lim_{z \to z_0} g(z) = L_2,
	\]
	then the following properties hold:
	\begin{enumerate}
		\item \( \lim_{z \to z_0} [f(z) + g(z)] = L_1 + L_2 \)
		\item \( \lim_{z \to z_0} [f(z) - g(z)] = L_1 - L_2 \)
		\item \( \lim_{z \to z_0} [f(z) g(z)] = L_1 L_2 \)
		\item If \( L_2 \neq 0 \), then \( \lim_{z \to z_0} \left[\frac{f(z)}{g(z)}\right] = \frac{L_1}{L_2} \)
	\end{enumerate}
\end{theorem}

\vfill

\clearpage

\begin{heading}
	Some Important Techniques for Finding the Limit of $\frac{P(z)}{Q(z)}$
\end{heading}

\vspace{10pt}

\begin{theorem}[Limit of Polynomials]
	If $f(z)$ is a polynomials in $z$, then the limit $\lim_{z \to z_0} f(z)=f(z_0)$ for any $z_0 \in \C$.
\end{theorem}


\vspace{10pt}

\begin{theorem}[Something / Non-zero]
	If $f(z)=\frac{P(z)}{Q(z)}$ where $P(z)$ and $Q(z)$ are polynomials, and if $Q(z_0) \neq 0$, then the limit $\lim_{z \to z_0} f(z)=\frac{P(z_0)}{Q(z_0)}$.
\end{theorem}


\vspace{10pt}

\begin{theorem}[Non-zero / Zero]
	If $f(z)=\frac{P(z)}{Q(z)}$ where $P(z)$ and $Q(z)$ are polynomials, and if $Q(z_0)=0$ while $P(z_0)\neq 0$, then the limit $\lim_{z \to z_0} f(z)$ does not exist.
\end{theorem}


\vspace{10pt}

\begin{theorem}[Zero / Zero]
	If $f(z)=\frac{P(z)}{Q(z)}$ where $P(z)$ and $Q(z)$ are polynomials, and if $Q(z_0)=0$ also $P(z_0) = 0$, then the limit $\lim_{z \to z_0} f(z)$ may exist. In this case, we can try to simplify $f(z)$ by factoring both $P(z)$ and $Q(z)$ and then canceling out the common factors. After simplification, we can re-evaluate the limit.
\end{theorem}

\clearpage

\begin{problem}
	Find the limit, if it exists:
	\[ \lim_{z \to 1+i} z^2 - 5z + 10 \]	
\end{problem}

\vfill

\begin{problem}
	Find the limit, if it exists:
	\[
	\lim_{z \to \tfrac{i}{2}} 
	\frac{(2z - 3)(4z + i)}{(iz - 1)^2}
	\]
\end{problem}

\vfill

\clearpage

\begin{problem}
	Find the limit, if it exists:
	\[ \lim_{z \to i} \frac{z^2-1}{z^6+1} \]	
\end{problem}

\vfill

\begin{problem}
	Find the limit, if it exists:
	\[
	\lim_{z \to 1 + i} 
	\left\{
	\frac{z - 1 - i}{z^2 - 2z + 2}
	\right\}^2
	\]	
\end{problem}

\vfill

\clearpage

\begin{problem}
	Let \( f(z) = \frac{z^2 + 1}{z - i} \). Prove that
	\[\lim_{h \to 0} \frac{f(z_0 + h) - f(z_0)}{h} 
	= \frac{7}{(3z_0 + 2)^2}, \quad 
	z_0 \ne -\tfrac{2}{3}.
	\]
\end{problem}

\clearpage

\begin{heading}
	L'Hospital's Rule
\end{heading}

\vspace{10pt}

\begin{theorem}[L'Hospital's Rule]
	Let \( f(z) \) and \( g(z) \) be functions that are differentiable on an open interval containing \( z_0 \), except possibly at \( z_0 \) itself. If
	\[
		\lim_{z \to z_0} f(z) = 0 \quad \text{and} \quad \lim_{z \to z_0} g(z) = 0,
	\]
	and if \( g'(z) \neq 0 \) for all \( z \) in the interval except possibly at \( z_0 \), then
	\[
		\lim_{z \to z_0} \frac{f(z)}{g(z)} = \lim_{z \to z_0} \frac{f'(z)}{g'(z)},
	\]
	provided the limit on the right side exists or is infinite.
\end{theorem}

\clearpage

\begin{problem}
	Find the limit, if it exists:
	\[ \lim_{z \to 0} \frac{z - \sin z}{z^3} \]
\end{problem}

\clearpage

\begin{problem}
	Find the limit, if it exists:
	\[ \lim_{z \to 0} \frac{z - \tan z z}{z^3} \]
\end{problem}

\clearpage

\begin{problem}
	Find the limit, if it exists:
	\[ \lim_{z \to e^{i\pi/3}} \left( z - e^{i\pi/3}\right) \frac{z}{z^3 + 1} \]
\end{problem}

\clearpage

\begin{problem}
	Find the limit, if it exists:
	\[ \lim_{z \to 0} \left( \frac{\sin z}{z} \right)^{\frac{1}{z^2}} \]
\end{problem}

\clearpage

\begin{problem}
	Find the limit, if it exists:
	\[ \lim_{z \to 0} \left( \frac{\sin z}{z} \right)^{\frac{\sin z}{z - \sin z}} \]
\end{problem}

\clearpage

\begin{problem}
	Find the limit, if it exists:
	\[ \lim_{z \to 0} \left( \frac{\tan z}{z} \right)^{\frac{1}{z^2}} \]
\end{problem}

\clearpage

\begin{problem}
	Find the limit, if it exists:
	\[ \lim_{z \to 0} \left( \sec z \right)^{\frac{1}{z^2}} \]
\end{problem}

\clearpage

\begin{problem}
	Find the limit, if it exists:
	\[ \lim_{z \to 0} \left( \sec z \right)^{\frac{1}{z^2}} \]
\end{problem}

\clearpage

\begin{heading}
	Continuity of Complex Functions
\end{heading}

\vspace{10pt}

\begin{definition}[Continuity of Complex Functions]
	A function \( f: D \to \C \) is said to be continuous at a point \( z_0 \in D \) if
	\[
		\lim_{z \to z_0} f(z) = f(z_0).
	\]
	If \( f \) is continuous at every point in its domain \( D \), then we say that \( f \) is continuous on \( D \).
\end{definition}

\vfill

\begin{theorem}[Note]
	$3$ conditions must be satisfied for a function \( f \) to be continuous at a point \( z_0 \):
	\begin{enumerate}
		\item \( f(z_0) \) is defined.
		\item \( \lim_{z \to z_0} f(z) \) exists.
		\item \( \lim_{z \to z_0} f(z) = f(z_0) \).
	\end{enumerate}
	If any of these conditions fail, then \( f \) is not continuous at \( z_0 \).
\end{theorem}

\clearpage

\begin{problem}
	Let
	\[
	f(z) = \frac{3z^4 - 2z^3 + 8z^2 - 2z + 5}{z - i}
	\]

	\begin{enumerate}
		\item Is \( f \) continuous at \( z = i \)?
		\item If \( f \) is not continuous at \( z = i \), redefine \( f(z) \) such that it becomes continuous at \( z = i \).
	\end{enumerate}
\end{problem}

\clearpage









\EndPage
\end{document}